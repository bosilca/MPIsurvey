\documentclass[11pt]{article}

\long\def\comment#1{}
\newcommand{\response}[2]{{\noindent\em #1}\\%
  {\bf Response:} #2}
\newcommand{\iresponse}[2]{{\item \em #1}\\%
  {\bf Response:} #2}

\begin{document}

\noindent
Dear Prof. \"{U}mit \c{C}ataly\"{u}rek,

\vspace{5mm}

Thanks for your comments. We have fixed all the issues based on the
reviews. Please take our response to each reviewer comment below.

\section*{Responses to Review 1}

We would like to thank reviewer 1 for reviewing our paper.

\response{
  Since the paper is a non-traditional one, there is no technical
  novelty in this work. The information provided is somewhat established
  connotations (e.g., MPI is a hard standard to implement), however
  there is some value to quantifying these beliefs with data. Hence,
  even though the margin of new information in this paper is slim, I
  still think the paper adds some value to the community.}
         {This paper is a survey paper based on a traditional
           questionnaire.  The method used in the paper is quite
           common and there is no new technical insights in this paper.
           However, the scale of the 
           questionnaire targeting MPI is unprecedented and very
           important in terms of
           the number of participants who are geographically
           wide-spread. As you know, the MPI standard is keep
           evolving. We believe that it is about the time to know how
           MPI is used and what is the problem when using MPI. Without
           having clear and right view of these, how can MPI widely be accepted?
           Connotations without having any backup data may lead to
           wrong directions.
           
         One of the findings of this paper is that many MPI users are
         sticking to the very basic functions, not knowing somewhat
         newly introduced functions. Again, MPI is keep evolving and
         getting more complex than ever. Will the newly introduced MPI
         functions be accepted by those who are still using old MPI functions?

         Most of the authors have been
         involved in MPI related projects of their own and have
         the experiences of attending MPI Forum meetings.  Like the
         all of the other MPI Forum members, all of us
         know MPI implementation is not easy at all but
         each of us knows only small number of MPI users near by.
         The sense of danger drove us to write this paper.

         Reviewers claimed that the findings of this paper is
         thin. This is because there is almost no contradictions to
         connotations.  The most crucial part of this paper is that
         these connotations are supported by the large-scale data.}

\response{
Overall, the paper needs a proofreading pass, and addressing the
following comments could improve their readability:
}
{We would like to appreciate your detailed comments and we fixed all
  proofreading issues.}

\begin{enumerate}
\iresponse{Please upfront define what (multiple) and (single) mean at the end
of each option (e.g., MPI Experience (single)). I had to guess the
meaning based on the context a lot later in the paper. It would be
better to add a sentence describing how to read these
options.}
          {We added the description at the very first appearance of it.}
  
\iresponse{Similarly, describing AD and ST in Section 3 on page 2 (last
sentence of the "Design" subsection) will not take up too much space
but are necessary for improving readability. These are defined in
Appendix as well, but if readers skip the Appendix, they will run the
risk of not understand the messages.}
{I am very sorry, I missed this comment. The ``AD'' and ``ST'' are
  appeared in Section 4.1.}

\iresponse{The text in the related work section can be converted into a table
and significantly improve that section's readability. E.g., it is hard
to remember what S, R, Q, etc., mean, and why does it matter what
their goals are? Explaining in terms of questions that S and R survey
categories cannot answer and where the strength of the Q category lies
will better explain the motivation of this paper. I am missing the
point of why there are so many categories and what their unique
contributions are.}
{I reorganized the whole problematic section.}

\iresponse{It is not clear what the authors mean when they say, "There is a
trade-off between the number of participants from each major
contributor and the number of the major contributorsin the cross-tab
arinalysis." Is "contributorsin" a typo? What is the trade-off? Which
cross-tab comparisons are being referred to? }
{I revised the text to reflect your comment.}

\iresponse{Define cross-tab analysis before using it. I do not know if it is a
standard term, and I assume most readers will not.}
{I added the definition of cross-tab analysis.}

\iresponse{"... may suggests ..." $\to$ "... may suggest ..." on page 8, the
  paragraph right below Figure 18.}
{I revised the text to reflect your comment.}

\iresponse{Page 8: "In same time, UK participants overwhelmingly learned MPI
from online sources, which usually translate by via practical
examples." $\to$ "At the same time,..... translated via .."}
{Updated.}

\iresponse{Page 11: Section 5.7 -- ".... contributorshave ....." $\to$
  "contributors have".}
{Updated.}

\end{enumerate}

\section*{Responses to Review 2}

We would like to thank reviewer 1 for reviewing our paper.

\response{The authors present the results and an in-depth discussion
  of an international MPI user survey performed between February 2019
  and some time in 2020. Unlike prior user surveys, the described
  survey was specifically designed to cover many countries and as such
  paint a diverse picture of the landscape of MPI users. The authors
  discuss each question thoroughly, however, in-depth correlation
  analysis between the answers of the different questions does not
  seem to have been done at large.}
 {
We developed a Python program to analyze answers. This program also
produces the heat map graphs of any possible combinations of two
questions in our questionnaire for cross-tab analysis. All heatmap
graphs were checked and only the significant ones are shown in the
paper. 
 }

\response{The paper is well structured and largely follows the
  structure of the questionnaire, yet the language could be improved
  in several places (see comments below). The chart visualizations are
  well done, yet often they focus on a single question and and provide
  only a single view on a specific question.}
{
           
}

\response{Especially the use of color in the bar charts does not feed
  a specific purpose, and more advanced info-vis chart types (e.g.,
  heatmaps) could improve the ink to information ratio.
  Also it's often not clear why a specific chart type was
  chosen for a specific question. Furthermore, information sometimes
  presented in tables could have been combined and presented with
  charts in a visual way.}
{I converted Table 6, 7 and 8 to graphs.}

\response{The questionaires used for the survey were specifically
  designed to attract many participants and the survey seems to be the
  largest of its kind in the near past. The methodology behind the
  questionaire (which topics, how many questions per topic, how to
  best formulate questions, etc.) should be discussed in a more
  structured way, explaining which underlying information was the
  target for a specific batch of questions.}
{
}

\response{Overall, the work presented provides a view on how MPI is
  used across the globe, yet many questions remain open. It presents a
  valuable source of information for the MPI community, yet should
  address the open questions and comments before publishing to unfold
  its full potential. Furthermore, it would be good if the authors
  also included an outlook sections on "lessons learned" from this
  survey and what should be done differently in future surveys of this
  kind.}
{
}

\begin{enumerate}
\iresponse{The authors used input on how to engineer the questionnaires
  from two social scientists. Next to the limitation of overall
  questions, which other factors shaped the design of the
  questionaires?}
{One of the important things when designing survey is to make
  questions as easy-to-answer as possible, so that participants can
  concentrate on the survey. Otherwise participants may give up.}

\iresponse{If there is a limitation of questions, how could one
  achieve a good coverage across all topics.}
{I am very sorry, I do not have any particular answer for this question.}

\iresponse{Why was I/O excluded, it seems that an opportunity was
  missed to check whether developers are aware of MPI's ability to
  parallelized their I/O (which at scale can easily become a
  bottleneck for the whole application).} 
{
Yes, we agree I/O can be a severe bottleneck in some applications. We
argued on this at the questionnaire design phase of this survey. One
reason for excluding MPI-IO comes from the limit of the number of
questions (30). Another reason is that the I/O issue often comes from
the underlying file system, not MPI-IO. We thought it is very
difficult for some MPI users to distinguish whether the I/O issue comes
from the underlying file system or MPI-IO itself. This can be true
when MPI users are domain experts but they lack system knowledge. And
we decided not to have MPI-IO related questions.
}

\iresponse{Was the elimination of counting lines of code a suggestion
  by the consulting professors? If so this might be an interesintg
  take away for future designs of similar surveys.}
{
No, we, not the social scientists, decided not to have the question
asking lines of code. Many numerical applications are large, but the
code to call MPI routines can be very small. Thus, the size of a
program may or may not correlate with the code size where calling MPI
routine. We could ask participants to count the lines only calling MPI
routines, but we wanted to make the questionnaire easy-to-answer by
eliminating such effort. 
}

\iresponse{The end date of the survey is given with "until recently",
  however, the publication data will be well in the future and the
  data when this is read even further. A similar end like "Month Year"
  should be given.}
{I put exact month/year of the most recent answer.}

\iresponse{The colors in the charts don't seem to be well suited for
  color-blind or color-impaired readers.}
{I changed the color set.}

\iresponse{Try to avoid meta language, explaining your thought
  process. For example: "One of the first questions we had to ask was
  how to reach a largely international ..."  $\to$ "To reach a largely
  international community, we ..." (more occasions of similar meta
  language throughout the document)]}
{blablabla}

\iresponse{Why is some data given in Table form and other in chart
  form. Try to convert as many tables into charts that actually
  contain numbers, or tables that contain charts (e.g., inline
  horizontal bar chart in the columns '\#Ans' and '\%'. A survey
  evaluation is all about information discovery and visualization, so
  it would be good to try to employ well seasoned chart types of the
  info-vis community.}
{Again, I converted Table 6, 7 and 8 to graphs.}

\iresponse{What is the trade off between number of participants and
  the number of contributors in the cross-tab analysis? This is
  unclear and should be made more explicit. What was limited? Why was
  the threshold of 50 chosen?}
{The larger the major contributors, the smaller the number of
  answers, this is what I meant. But, I agree, this statement was not
  clear and I changed the wording.  The threshold was chosen to
  include USA which plays very important role in the HPC field,
  therefor we thought USA must be included in the major contributors.}

\iresponse{Regarding the correlation of Figure 3, how could this
  ambiguity be avoided in a future survey?}
{We had expected to have more number of the AI choice, since AI is
  getting the big concern nowadays. You may think that the numerical
  app/lib dominates because this is a multi-answer question. However,
  when I checked the raw data, the choice of the numerical app/lib has
  334 answers, and the second largest is the combination of language and
  numerical app/lib (only 48). So, we think Fig. 3 is representing the
  participants profile. }

\iresponse{Figure placement should NOT be inline but rather floating
  top or bottom of a page. Text such as page 4 between Fig 4 and 5 are
  easily overlooked and don't provide a good page flow.}
{I updated the figure and table positions to place them at the top or
  bottom, not inline.}

\iresponse{It might be better to transform Table 4 into a
  double-column float to have longer lines for the questions.}
{It is a good idea, thank you. I converted the table.}

\iresponse{Which data is referenced by "In the other major
  contributors" (p.4)? "Europe:other"? This is ambiguous and should be
  clarified.}
{I modified the paragraph to clarify this.}

\iresponse{Fig. 7: A feature like PMPI is quite specific to a single
  community (tool bilders - with rare exceptions), so it seems awkward
  to compare its use with other features that resemble a
  specialization of well-used features (persistent)}
{I am sorry, I do not agree with you about PMPI. PMPI is defined in the
standard and is to provide a somewhat tricky but very strong mechanism
for users to investigate how MPI routine are called. And we wanted to
see how much popular among users PMPI is.}

\iresponse{Fig. 7: Could Q7C9 be correlated with Q17C9? If so this
  could explain why people of all skill levels use PMPI ... you just
  have more or less experienced tool developers here. (Such questions
  could be answered addressed by a more systemativ cross examination
  of the different questions).}
{Let me answer this comments, assuming Q3 asking self assessment MPI
  skill and Q17 asking using MPI aspects. I checked the Q3-Q17 heatmap
graph again, but the number of PMPI answers occupied very small
fraction, and I could not find any obvious correlation between them.}

\iresponse{With citing other books, tutorials and other online
  resources, maybe a correlation of "what is used" with "what is
  taught" can be found? This should be investigated. The claim: "The
  MPI standard maybe too complex and leads to limited understanding"
  (p.5) may not be the only conclusion that could be drawn from the
  survey data.}
{I checked the heatmaps of Q14-Q16 and Q14-Q17. The situation is very
  similar to the above Q3-Q17. Those heatmap graphs reflects the
  biased distribution of the answers and no significant correlations
  can be found. When I firstly checked the generated heatmap graphs, I
  was very disappointed because only some of them have significant
  correlations which are shown in the paper.}

\iresponse{The authors claim that by having a fixed boundaries and
  conditions (p.6) leads to static MPI applications. However, the
  boundaries are not what makes a program dynamic. A dynamic need for
  computing resources (and proper re-balancing of such) can be present
  in such static scientific domains. What is not investigated here is
  that the operation of HPC systems (batch scheduling) often prohibits
  the dynamic allocation of new resources for a running job. However,
  with the survey data this can probably not be
  investigated. (Potential question for a future survey?)} 
{I agree with you about the usage of dynamic process creation and the
  operation policy does not allow users to create processes dynamically.
  However, in the current text does not deny the usage you pointed
  out.}

\iresponse{Also for future surveys: Maybe users are not aware that
  SINGLE is the thread level that MPI\_Init uses. Does this correlate
  with answers of "MPI-only" applications?}
{I checked the cross-tab graph between Q17 asking thread model and
  MPI+X. As the number of answering MPI only users ('No' in our
  survey) is not large and we could not see any strong correlation
  between them. }

\iresponse{The percentage numbers in Table 8 don't add up to
  100\%. What do they refer to?}
{This is because there are many other multiple-answer combinations and
only the top 7 combinations are listed in this table.}

\iresponse{Page 8: "Portability, which translates into maintaining
  backward compatibility across versions, ..." ... Portability is much
  more than that and would in my mind first and foremost be a
  portability across different MPI libraries and Hardware using the
  same MPI version.}
{I am very sorry on this point. This is my fault of using the word
  'portability' in Q28, since we are talking on (backward) compatibility
here. We changed the wording here.}

\iresponse{Why is is interesting that the same number of people have
  checked "read the full standard" and "never read the standard"?
  Also, Figure 20 does not support this claim.}
{UK is very interesting because it has both extremes, no other
  countries exhibits the similar situation. Yes, the wording was wrong
and I changed the text.}

\iresponse{The authors remark that many users did not know about
  Endpoints (a proposal never officially released with MPI and
  currently abandoned?), yet maybe the questions should have been
  rephrased not to ask about a specific solution but more on what
  Endpoints was trying to address: "Do you whish MPI to be
  thread-aware?" or "Do you wish an interface that helps with
  performant multi-threaded interaction with the MPI library?"}
{The endpoint topic was very actively discussed at the MPI Forum
  meetings at that time. So we wanted to know how many users are
  interested in or have ever heard of that term regardless how much
  users understand it correctly. We are a bit
  surprised because the percentage of endpoints are larger than that
  we thought. }

\iresponse{The authors identify a focus on MPI+CUDA as
  "interesting". Why is this interesting? Which hypothesis could one
  derive that would/should be tested by a future survey? (The authors
  should extend the discussion here)}
{I am sorry we can not go into more details here because none of us
  has any strong connections with the Russian HPC community. }

\end{enumerate}

\response{Syntax/Grammar/Wording:}
{We fixed everything pointed out.}

\comment{
=======================
p.1, left col.:
---------------
would have the same $\to$ would have had the same
a single question $\to$ only a single question
many diverse community $\to$ many diverse communities

p.1, right col.:
feature differences of MPI users $\to$
different use of MPI features among users across ...

No comma between the two consulting professors
... Illinois Univ. and Prof. ....

p.2, left col.:
---------------
user a colon and inline enumeration:
... three survey classes: (1) questionaire surveys ...; (2)
application-oriented statistical surveyes ....; and (3)
application-oriented statistical surveys .... 

questionaire survey $\to$ questionaire surveys

are become $\to$ are becoming

Use the publications authors (Foo and Bar, Foo et al., etc.) as
subject no the reference itself ...: 
[6] statistically investigated $\to$ Laguna et al. [6] statistically
investigated ... 

features translates $\to$ features translate

DOE Mini-apps $\to$ DOE mini-apps

2 studies $\to$ two studies

target of S is MPI programs $\to$ target of S are MPI programs

target of R is MPI programs $\to$ target of R are MPI programs

The social scientists $\to$ Name them explicitly again here:
Poole and Ahmed, our consulting social scientists, ...

limited around 30 $\to$ limited to around 30

p.2, right col.:
----------------

China has $\to$ China had

p.3, left col.:
---------------
distibuting fliers $\to$ distributing flyers

contributorsin $\to$ contributors in

p.3, right col.:
----------------
xxxxxx Figure 2, legend: spell out options. It's not immediately apparent
what the options mean. 

2 major contributors $\to$ two major contributors
(numbers below and including 'twelve' should be spelled out)

XXXXXX Before going into the details ... $\to$ avoid meta language

community based word-of-the-mouth $\to$ community-based word-of-mouth

XXXXXX "HPC-centered" is bleeding into the margin

of self-evaluation $\to$ of the self-evaluation

participants MPI skill $\to$ participants' MPI skill

p.4, left col.:
---------------
do no rank $\to$ do not rank

Japan followed closely has $\to$ Japan followed closely and has

p.4, right col.:
----------------
was release in $\to$ was released in

p.5, left col.:
---------------
it goes against the expected outcome $\to$
it contradicts the expected outcome

p.6, left col.:
---------------
creation spreads $\to$ creation involves

"I don't know" vs. "No idea" $\to$ use "I don't know" consistently

p.7, left col.:
---------------

dominates in all $\to$ dominate for all

users community $\to$ user community

have experienced writing $\to$ have experience with writing

p.7, right col.:
----------------
XXXXXX Sentence structure problem:
... that the approximately 3/4 of the target programs ...

p.8, left col.:
---------------
the MPI alternative $\to$ an MPI alternative

by better alternative $\to$ by a better alternative

middle group who chose $\to$ middle group that chose

excepting Russia $\to$ except Russia

p.8, right col.:
----------------
accept incompatibility $\to$ accept incompatibilities

in exchange of $\to$ in exchange for

In same time $\to$ At the same time

which usually translate by $\to$ which usually translates to

p.9, left col.:
---------------
Improve sentence structure:
There can be seen a weak correlation, ...

(basically those who ... $\to$ (those who

in text boxes lead to $\to$ in text boxes leads to

p.9, right col.:
----------------
would have been led $\to$ would have led

improve sentence structure:
pinpointing to how time consuming master MPI is.

master MPI $\to$ mastering MPI

participants complaints $\to$ participants complaint

Remove comma:
Q3 (Fig.4), highlight ... $\to$ Q3 (Fig.4) hightlight

p.10, left col.:
---------------

to be lots of room to tune MPI programs $\to$
to be a lots of potential for tuning MPI programs

performance become $\to$ performance becomes

p.11, left col.:
---------------
This question tackles $\to$ This question addresses

what we seen $\to$ what we have seen

contributorshave $\to$ contributors have

"by some standards" is unclear. By which standards?

p.11, right col.:
-----------------
The high concern $\to$ The high focus on

sound promising $\to$ sounds promising

future of Japanese $\to$ future of the Japanese

not new being $\to$ not new, being

widely accepted $\to$ widely adopted

p.12, left col.:
----------------
Another reason of $\to$ Another reason for

Hard to understand:
"There is no way for library functions ..." $\to$
"There is no way for the implementation of low-level communication procedures
to know whether they are employed as part of a higher-level data exchange
pattern."

receive[15] $\to$ receive [15]

widely accepted $\to$ widely adopted

If MPI Forum $\to$ If the MPI Forum

agrees ... for $\to$ agrees ... on

then MPI Forum $\to$ then the MPI Forum

p.12, right col.:
is, at least perceived as a ...$\to$
is, at least perceived as, a ...
}

\section*{Responses to Review 2}

\response{The results from this survey does not really provide any
  new findings that were not already discovered with other smaller
  surveys other than minor differences in user's experience based on
  specific country or region. One of the outcome of this survey is a
  suggestion to the MPI Forum to improve the educational effort to
  bridge the gap between the features included in the standard to
  their corresponding usage and familiarity.}
         {This paper is a survey paper based on a traditional
           questionnaire.  The method used in the paper is quite
           common and there is no new technical insights in this paper.
           However, the scale of the 
           questionnaire targeting MPI is unprecedented and hopefully
           results in more accurate than the others in terms of
           statistics.
           Therefor the findings of ours and theirs look similar, but
           the nature of the underlying data is quite different.
}

\response{But, it is not clear how the MPI Forum can address this
  issue, since the goal of any standardization body should be define
  the standard. The MPI Forum already supports several activities such
  as BoFs and there are a large number of books, free tutorials, and
  other material available at most HPC centers, in addition to
  workshops and tutorials dedicated to the use of MPI.}
{Most of the authors have the experiences of attending MPI Forum
  meetings.  They are having BOF meetings at SC, and many MPI Forum
  members have the experiences of teaching MPI, lecturing MPI, and/or
  writing MPI books. However, our survey reveals that many MPI users
  are willing to have another form of MPI information, despite those
  efforts. As the quote at Sec. 6.2 ``most of such web pages are
  out-dated and not kept in sych with today’s web standards.'' tells,
  most of the web pages and tutorial slides available online are
  outdated. So many MPI users are longing to have consistent,
  up-to-dated, well-structured teaching materials. This could only be done
  by MPI Forum itself. 
}

\response{Data in Brief (optional): We invite you to convert your
  supplementary data (or a part of it) into an additional journal
  publication in Data in Brief, a multi-disciplinary open access
  journal. Data in Brief articles are a fantastic way to describe
  supplementary data and associated metadata, or full raw datasets
  deposited in an external repository, which are otherwise
  unnoticed. A Data in Brief article (which will be reviewed,
  formatted, indexed, and given a DOI) will make your data easier to
  find, reproduce, and cite.}
{We are happy to do this.}

\end{document}
