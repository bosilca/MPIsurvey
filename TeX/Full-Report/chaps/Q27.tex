
\subsection{Answers}
\begin{table}[htb]%
\begin{center}%
\caption{Q27: What MPI feature(s) are NOT useful for you application?}%
\label{tab:Q27-ans}%
\begin{tabular}{l|l|r}%
\hline%
Choice & Abbrv. & \# Answers \\%
\hline%
There are no unnecessary features & No & 309 (41.5\%) \\%
Dynamic process creation & Dynamic process & 274 (36.8\%) \\%
Process topologies & Topologies & 126 (16.9\%) \\%
One-sided communication & One-sided & 103 (13.8\%) \\%
Error handlers & Error & 72 (9.7\%) \\%
Communicator and group management & Communicator & 63 (8.5\%) \\%
Datatypes & Datatypes & 58 (7.8\%) \\%
Collective operations & Collectives & 11 (1.5\%) \\%
other & - & 20 (2.7\%) \\%
\hline%
\multicolumn{2}{c}{total} & 1036 (745)\\%
\hline%
\end{tabular}%
\end{center}%
\end{table}%


The question of unnecessary features for the user’s applications relates on how
the usefulness of each chapter of the MPI standards is perceived. Unsurprisingly, the
least unwanted feature is the collective operation: this feature is seen as
very important and very useful. Datatypes, communicators and group management
as well as  error handlers received less than 10\% of the answers and are
therefore seen as very useful characteristics of the standard. On the other
side, dynamic process creation received 29\% of the answers. This means that
such feature is not used in many applications and that more generally, MPI is seen
as a static programming model where the number of processes does not evolve
during the program execution. To a least extent, process topologies and
one-sided communication received respectively 12\% and 10\% of the answers. This
is not very surprising as these features relatively specific to special
use-cases. 

It is a bit odd that the most people say the collective ops is
usefull, however, 61 people feel that communicatos (and group) ops are
useless. Does this mean many MPI users do not create their own communicatos?

\subsection{List of other answers}
\begin{itemize}
\item Europe:France: things I don't know like Dynamic process creation etc
\item Europe:Germany: Don't know
\item Europe:Germany: Haven't tried all. So can not comment
\item Europe:Germany: I am not aware of unncessary features, but maybe my overview is not large enough.
\item Europe:Germany: I am not sure because I don't have the full overview
\item Europe:Germany: I don't know
\item Europe:Germany: I use only a small part of the MPI features
\item Europe:Germany: do not know
\item Europe:Germany: the performance of some features needs improvement
\item Europe:others: I do not know
\item Europe:others: I do not know :)
\item Europe:others: I don't think I need them. But they are useful for domain decomposition.
\item Europe:others: no clue
\item Japan: I am not sure
\item Japan: I don't know.
\item North America: Tag matching is not needed most of the time. I would exchange it in return for more performance.
\item Russia: Don't know (too little experience with MPI)
\item Russia: I'm not sure
\item USA: I/O
\item USA: all of the above are useful

\end{itemize}

\begin{figure}[htb]
\begin{center}
\includegraphics[width=10cm]{../pdfs/Q27.pdf}
\caption{Simple analysis: Q27}
\label{fig:Q27}
\end{center}
\end{figure}
