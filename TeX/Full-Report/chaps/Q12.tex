
\subsection{Answers}
\begin{table}[htb]%
\begin{center}%
\caption{Q12: Which MPI implementations do you use?}%
\label{tab:Q12-ans}%
\begin{tabular}{l|l|r}%
\hline%
Choice & Abbrv. & \# Answers \\%
\hline%
Open MPI & OMPI & 723 (85.4\%) \\%
Intel MPI & Intel & 522 (61.6\%) \\%
MPICH & MPICH & 470 (55.5\%) \\%
MVAPICH & MVA & 175 (20.7\%) \\%
Cray MPI & Cray & 145 (17.1\%) \\%
IBM MPI (BG/Q, PE, Spectrum) & IBM & 92 (10.9\%) \\%
Fujistu MPI & Fujitsu & 29 (3.4\%) \\%
MS MPI & MS & 27 (3.2\%) \\%
HPE MPI & HPE & 23 (2.7\%) \\%
NEC MPI & NEC & 23 (2.7\%) \\%
I do not know & No idea & 10 (1.2\%) \\%
MPC MPI & MPC & 8 (0.9\%) \\%
Sunway MPI & Sunway & 5 (0.6\%) \\%
Tianhe MPI & Tianhe & 3 (0.4\%) \\%
other & - & 27 (3.2\%) \\%
\hline%
\multicolumn{2}{c}{total} & 2282 (847)\\%
\hline%
\end{tabular}%
\end{center}%
\end{table}%


\begin{figure}[htb]
\begin{center}
\includegraphics[width=10cm]{../pdfs/Q12.pdf}
\caption{Breakdown of MPI implementation usages per location}
\label{fig:Q12}
\end{center}
\end{figure}

\begin{figure}[htb]
\begin{center}
\includegraphics[width=14cm]{../pdfs/Q12-mans.pdf}
\caption{Multiple Answer: Q12}
\label{fig:Q12-mans}
\end{center}
\end{figure}

Looking at Table~\ref{tab:Q12-ans} from the perspective of the origin of the MPI
implementation, either public domain or proprietary (vendor) provided MPI, we can
notice that about 60\% of the answers are using open source MPI
implementations (Open MPI, MPICH and MVAPICH), not provided by a
vendor.

Looking more precisely at the 27 other MPI implementations specifically
mentioned in the answers, points to the expected dichotomy of vendor's imposed
or delivered MPI implementations, as well as to some lesser known MPI
implementations with an existing, but usually geographically constrained user
base. For more information regarding geographical distribution of MPI
implementations, see Section~\ref{sec:disc:impl}.
For the sake of completeness it should be noted that some MPI implementations
were mentioned at least once (Sun~MPI, Adaptive~MPI, PGI~MPI, Platform~MPI) as
well as a Python API for MPI (mpi4py).

% \mycomment[AH]{I do not think they are geographically constrained, but
% vendors force users to use their MPI. For example, BG/* users have no
% choice but using IBM MPI, the K users must use Fujistu MPI.}.

% \mycomment[AH]{It would be nice to categorize MPI implementations into
% two; proprietary (vendor) MPI and public domain MPI}.

In Table~\ref{tab:Q12-mans}, Open MPI is the most frequently used MPI
implementation in the standalone way.  Although the K computer and its
successors only support Fujitsu MPI, most machines support several
MPI implementations. It would be interesting to know how users choose
an MPI implementation among them.

\clearpage%
{\footnotesize\begin{landscape}%
\begin{longtable}[htb]{r|c|c|c|c|c|c|c|c|c|c}%
\caption{Q12: Which MPI implementations do you use?}%
\label{tab:Q12-mans} \\%
\hline%
Multi-Answer & overall & FR & GR & IT & UK & eu & JP & RU & US & others \\
 \hline%
\endfirsthead%
\multicolumn{11}{r}{(continued from the previous page)}\\%
\hline%
Multi-Answer & overall & FR & GR & IT & UK & eu & JP & RU & US & others \\
 \hline%
\endhead%
\hline%
(total) & 847 & 123 & 159 & 57 & 67 & 144 & 64 & 93 & 58 & 82 \\%
\hline%
\multicolumn{11}{r}{(continue to the next page)}\\%
\endfoot%
\hline%
(total) & 847 & 123 & 159 & 57 & 67 & 144 & 64 & 93 & 58 & 82 \\%
\hline%
\endlastfoot%
\hline%
{OMPI, Intel} & 126 & 27 & 25 & 17 & 6 & 21 & 3 & 17 & 2 & 8 \\%
{MPICH, OMPI, Intel} & 93 & 19 & 19 & 6 & 13 & 18 & 2 & 7 & 4 & 5 \\%
{OMPI} & 91 & 9 & 13 & 10 & 4 & 18 & 5 & 13 & 6 & 13 \\%
{MPICH, OMPI} & 69 & 13 & 9 & 4 & 4 & 13 & 3 & 10 & 4 & 9 \\%
{MPICH, OMPI, MVA, Intel} & 35 & 3 & 7 & 4 & 2 & 5 & 5 & 4 & 2 & 3 \\%
{MPICH, OMPI, MVA} & 26 & 3 & 4 & 1 & 8 & 3 & 0 & 4 & 3 & 0 \\%
{Intel} & 23 & 2 & 6 & 2 & 0 & 3 & 0 & 6 & 1 & 3 \\%
{MPICH, Intel} & 20 & 1 & 3 & 2 & 1 & 1 & 2 & 7 & 0 & 3 \\%
{OMPI, MVA, Intel} & 19 & 2 & 2 & 1 & 2 & 3 & 2 & 0 & 2 & 5 \\%
{MPICH} & 18 & 1 & 3 & 1 & 1 & 1 & 1 & 4 & 1 & 5 \\%
{MPICH, OMPI, Intel, Cray} & 18 & 1 & 4 & 0 & 3 & 5 & 1 & 0 & 4 & 0 \\%
{MPICH, OMPI, Intel, IBM} & 16 & 6 & 5 & 1 & 0 & 2 & 0 & 2 & 0 & 0 \\%
{OMPI, Intel, Cray} & 14 & 0 & 4 & 2 & 3 & 3 & 0 & 0 & 2 & 0 \\%
{MPICH, OMPI, MVA, Intel, Cray} & 14 & 0 & 2 & 0 & 4 & 4 & 1 & 0 & 3 & 0 \\%
{MPICH, OMPI, Cray} & 14 & 0 & 3 & 1 & 1 & 9 & 0 & 0 & 0 & 0 \\%
{MPICH, OMPI, Intel, Cray, IBM} & 10 & 1 & 2 & 0 & 0 & 3 & 0 & 0 & 2 & 2 \\%
{No idea} & 9 & 3 & 1 & 0 & 0 & 1 & 1 & 1 & 0 & 2 \\%
{MPICH, OMPI, MVA, Cray} & 8 & 0 & 1 & 0 & 1 & 6 & 0 & 0 & 0 & 0 \\%
{MPICH, OMPI, MVA, Intel, IBM} & 7 & 2 & 3 & 0 & 0 & 1 & 0 & 1 & 0 & 0 \\%
{OMPI, IBM} & 7 & 2 & 0 & 0 & 0 & 0 & 0 & 2 & 3 & 0 \\%
{OMPI, Cray} & 7 & 0 & 3 & 0 & 0 & 2 & 0 & 0 & 0 & 2 \\%
{MPICH, OMPI, MVA, Intel, Cray, IBM} & 7 & 1 & 1 & 2 & 1 & 0 & 0 & 0 & 1 & 1 \\%
{MPICH, OMPI, Intel, Fujitsu} & 6 & 0 & 1 & 0 & 0 & 0 & 5 & 0 & 0 & 0 \\%
{MPICH, OMPI, Intel, MS} & 6 & 1 & 3 & 0 & 0 & 0 & 0 & 2 & 0 & 0 \\%
{MVA} & 6 & 0 & 0 & 0 & 0 & 0 & 3 & 0 & 1 & 2 \\%
{OMPI, Intel, IBM} & 5 & 1 & 0 & 0 & 1 & 1 & 0 & 0 & 0 & 2 \\%
{OMPI, Intel, HPE} & 5 & 1 & 1 & 0 & 0 & 1 & 2 & 0 & 0 & 0 \\%
{MPICH, OMPI, IBM} & 5 & 0 & 2 & 1 & 0 & 0 & 0 & 1 & 1 & 0 \\%
{OMPI, Intel, MS} & 5 & 0 & 1 & 0 & 0 & 0 & 0 & 3 & 0 & 1 \\%
{MPICH, Cray} & 5 & 0 & 0 & 0 & 3 & 1 & 0 & 0 & 1 & 0 \\%
{MPICH, OMPI, MS} & 4 & 0 & 0 & 0 & 1 & 0 & 1 & 2 & 0 & 0 \\%
{MPICH, OMPI, Intel, Cray, NEC} & 4 & 0 & 2 & 0 & 0 & 2 & 0 & 0 & 0 & 0 \\%
{MPICH, OMPI, Intel, HPE} & 4 & 1 & 1 & 0 & 1 & 0 & 1 & 0 & 0 & 0 \\%
{ParaStation MPI} & 4 & 0 & 4 & 0 & 0 & 0 & 0 & 0 & 0 & 0 \\%
{MPICH, OMPI, Cray, IBM} & 4 & 0 & 0 & 0 & 1 & 0 & 0 & 0 & 2 & 1 \\%
{MPICH, MS} & 3 & 0 & 0 & 0 & 0 & 0 & 0 & 2 & 0 & 1 \\%
{OMPI, Intel, Fujitsu} & 3 & 0 & 0 & 0 & 0 & 0 & 3 & 0 & 0 & 0 \\%
{MPICH, MVA, Intel} & 3 & 0 & 0 & 1 & 0 & 0 & 0 & 0 & 0 & 2 \\%
{MPICH, OMPI, Fujitsu} & 3 & 0 & 0 & 0 & 0 & 0 & 3 & 0 & 0 & 0 \\%
{MPICH, OMPI, MVA, Intel, Cray, HPE} & 3 & 0 & 0 & 0 & 3 & 0 & 0 & 0 & 0 & 0 \\%
{MPICH, OMPI, Fujitsu, NEC} & 3 & 0 & 0 & 0 & 0 & 0 & 3 & 0 & 0 & 0 \\%
{MPICH, MVA} & 3 & 0 & 0 & 0 & 0 & 1 & 0 & 0 & 1 & 1 \\%
{MPICH, MVA, Cray} & 3 & 0 & 0 & 0 & 0 & 2 & 0 & 0 & 1 & 0 \\%
{OMPI, MVA} & 3 & 0 & 1 & 0 & 0 & 0 & 0 & 1 & 0 & 1 \\%
{MPICH, OMPI, Intel, NEC} & 3 & 0 & 1 & 0 & 0 & 1 & 1 & 0 & 0 & 0 \\%
{OMPI, Intel, Cray, IBM} & 2 & 1 & 0 & 0 & 0 & 1 & 0 & 0 & 0 & 0 \\%
{OMPI, HPE} & 2 & 1 & 0 & 0 & 0 & 0 & 0 & 0 & 0 & 1 \\%
{MPICH, OMPI, MVA, Intel, Cray, IBM, HPE} & 2 & 1 & 0 & 0 & 1 & 0 & 0 & 0 & 0 & 0 \\%
{MPICH, OMPI, MPC} & 2 & 2 & 0 & 0 & 0 & 0 & 0 & 0 & 0 & 0 \\%
{OMPI, MVA, Cray} & 2 & 0 & 0 & 0 & 0 & 2 & 0 & 0 & 0 & 0 \\%
{MPICH, MVA, Intel, Cray} & 2 & 0 & 0 & 0 & 0 & 1 & 0 & 0 & 1 & 0 \\%
{Intel, Fujitsu} & 2 & 0 & 0 & 0 & 0 & 0 & 2 & 0 & 0 & 0 \\%
{OMPI, MVA, Intel, IBM} & 2 & 0 & 1 & 0 & 0 & 1 & 0 & 0 & 0 & 0 \\%
{OMPI, Intel, IBM, HPE} & 2 & 1 & 1 & 0 & 0 & 0 & 0 & 0 & 0 & 0 \\%
{MPICH, OMPI, MVA, Fujitsu} & 2 & 0 & 0 & 0 & 0 & 0 & 2 & 0 & 0 & 0 \\%
{Intel, IBM} & 2 & 0 & 1 & 0 & 0 & 1 & 0 & 0 & 0 & 0 \\%
{MPICH, Intel, IBM} & 2 & 1 & 1 & 0 & 0 & 0 & 0 & 0 & 0 & 0 \\%
{MPICH, OMPI, MVA, Intel, Cray, NEC} & 2 & 0 & 1 & 0 & 0 & 0 & 0 & 0 & 0 & 1 \\%
{OMPI, MVA, Intel, Fujitsu} & 2 & 0 & 0 & 0 & 0 & 0 & 2 & 0 & 0 & 0 \\%
{Intel, Cray} & 2 & 0 & 0 & 0 & 0 & 0 & 2 & 0 & 0 & 0 \\%
{MPICH, OMPI, MVA, Intel, Fujitsu} & 2 & 0 & 0 & 0 & 0 & 0 & 2 & 0 & 0 & 0 \\%
{OMPI, Fujitsu} & 2 & 0 & 0 & 0 & 0 & 0 & 2 & 0 & 0 & 0 \\%
{MPICH, Intel, Cray} & 1 & 0 & 0 & 0 & 0 & 0 & 0 & 1 & 0 & 0 \\%
{MVA, Intel} & 1 & 0 & 0 & 0 & 0 & 0 & 0 & 0 & 1 & 0 \\%
{MPICH, OMPI, Intel, Tianhe, Sunway} & 1 & 0 & 0 & 0 & 0 & 0 & 0 & 0 & 0 & 1 \\%
{OMPI, Intel, Cray, NEC} & 1 & 0 & 1 & 0 & 0 & 0 & 0 & 0 & 0 & 0 \\%
{MPICH, OMPI, MVA, Intel, IBM, ParaStation MPI} & 1 & 0 & 1 & 0 & 0 & 0 & 0 & 0 & 0 & 0 \\%
{MPICH, OMPI, Intel, Cray, SGI MPI} & 1 & 0 & 0 & 0 & 0 & 0 & 0 & 1 & 0 & 0 \\%
{MPICH, OMPI, Cray, MS} & 1 & 0 & 0 & 0 & 0 & 1 & 0 & 0 & 0 & 0 \\%
{MPICH, OMPI, MVA, IBM} & 1 & 0 & 0 & 0 & 0 & 0 & 0 & 0 & 1 & 0 \\%
{OMPI, MVA, Intel, Cray, NEC} & 1 & 0 & 1 & 0 & 0 & 0 & 0 & 0 & 0 & 0 \\%
{Fujitsu} & 1 & 0 & 0 & 0 & 0 & 0 & 1 & 0 & 0 & 0 \\%
{OMPI, Intel, Cray, IBM, HPE} & 1 & 0 & 0 & 0 & 1 & 0 & 0 & 0 & 0 & 0 \\%
{MPICH, OMPI, Intel, IBM, SGI MPI} & 1 & 1 & 0 & 0 & 0 & 0 & 0 & 0 & 0 & 0 \\%
{OMPI, Intel, Cray, IBM, MS} & 1 & 0 & 0 & 0 & 0 & 0 & 0 & 0 & 1 & 0 \\%
{OMPI, Intel, MPC} & 1 & 1 & 0 & 0 & 0 & 0 & 0 & 0 & 0 & 0 \\%
{MPICH, OMPI, HPE} & 1 & 0 & 0 & 0 & 0 & 0 & 1 & 0 & 0 & 0 \\%
{MPICH, Intel, Cray, SGI MPT} & 1 & 0 & 0 & 0 & 1 & 0 & 0 & 0 & 0 & 0 \\%
{MS} & 1 & 0 & 0 & 0 & 0 & 0 & 0 & 0 & 0 & 1 \\%
{OMPI, MVA, Intel, HPE} & 1 & 0 & 0 & 0 & 0 & 0 & 1 & 0 & 0 & 0 \\%
{MPICH, OMPI, Intel, Tianhe} & 1 & 0 & 0 & 0 & 0 & 0 & 0 & 0 & 0 & 1 \\%
{MPICH, OMPI, MVA, Intel, Cray, IBM, Fujitsu} & 1 & 0 & 0 & 0 & 0 & 1 & 0 & 0 & 0 & 0 \\%
{OMPI, MPC} & 1 & 1 & 0 & 0 & 0 & 0 & 0 & 0 & 0 & 0 \\%
{MPICH, OMPI, Intel, BULL MPI} & 1 & 1 & 0 & 0 & 0 & 0 & 0 & 0 & 0 & 0 \\%
{OMPI, Intel, IBM, NEC} & 1 & 0 & 1 & 0 & 0 & 0 & 0 & 0 & 0 & 0 \\%
{Intel, Cray, IBM} & 1 & 0 & 1 & 0 & 0 & 0 & 0 & 0 & 0 & 0 \\%
{OMPI, MVA, Intel, NEC} & 1 & 0 & 1 & 0 & 0 & 0 & 0 & 0 & 0 & 0 \\%
{MPICH, OMPI, MadMPI} & 1 & 1 & 0 & 0 & 0 & 0 & 0 & 0 & 0 & 0 \\%
{MPICH, OMPI, MVA, Intel, Cray, MS} & 1 & 0 & 0 & 0 & 0 & 1 & 0 & 0 & 0 & 0 \\%
{MPICH, OMPI, Intel, ParaStation MPI} & 1 & 0 & 1 & 0 & 0 & 0 & 0 & 0 & 0 & 0 \\%
{MPICH, OMPI, MVA, Intel, Cray, IBM, NEC} & 1 & 0 & 1 & 0 & 0 & 0 & 0 & 0 & 0 & 0 \\%
{MPICH, OMPI, BullMPI} & 1 & 1 & 0 & 0 & 0 & 0 & 0 & 0 & 0 & 0 \\%
{MPICH, OMPI, MVA, Cray, MS} & 1 & 0 & 0 & 0 & 0 & 1 & 0 & 0 & 0 & 0 \\%
{MPICH, OMPI, No idea} & 1 & 0 & 1 & 0 & 0 & 0 & 0 & 0 & 0 & 0 \\%
{MPICH, OMPI, MVA, Intel, MPC} & 1 & 1 & 0 & 0 & 0 & 0 & 0 & 0 & 0 & 0 \\%
{Sunway} & 1 & 0 & 0 & 0 & 0 & 0 & 0 & 0 & 0 & 1 \\%
{OMPI, MVA, Intel, Cray, IBM} & 1 & 0 & 0 & 0 & 0 & 1 & 0 & 0 & 0 & 0 \\%
{OMPI, Intel, MPC, madMPI} & 1 & 1 & 0 & 0 & 0 & 0 & 0 & 0 & 0 & 0 \\%
{MPICH, MVA, Cray, IBM} & 1 & 0 & 0 & 0 & 0 & 0 & 0 & 0 & 1 & 0 \\%
{MPICH, OMPI, MVA, Intel, MPC, MadMPI} & 1 & 1 & 0 & 0 & 0 & 0 & 0 & 0 & 0 & 0 \\%
{MPICH, OMPI, Intel, IBM, BullX-MPI} & 1 & 1 & 0 & 0 & 0 & 0 & 0 & 0 & 0 & 0 \\%
{MPICH, OMPI, MVA, Intel, ParaStation MPI} & 1 & 0 & 1 & 0 & 0 & 0 & 0 & 0 & 0 & 0 \\%
{Intel, ParaStation MPI} & 1 & 0 & 1 & 0 & 0 & 0 & 0 & 0 & 0 & 0 \\%
{Intel, Fujitsu, NEC, SGI MPI} & 1 & 0 & 0 & 0 & 0 & 0 & 1 & 0 & 0 & 0 \\%
{MPICH, Intel, Sunway} & 1 & 0 & 0 & 0 & 0 & 0 & 0 & 0 & 0 & 1 \\%
{Intel, MS} & 1 & 0 & 1 & 0 & 0 & 0 & 0 & 0 & 0 & 0 \\%
{MPICH, OMPI, Intel, IBM, MS} & 1 & 0 & 0 & 0 & 0 & 0 & 0 & 1 & 0 & 0 \\%
{MPICH, OMPI, NEC} & 1 & 1 & 0 & 0 & 0 & 0 & 0 & 0 & 0 & 0 \\%
{OMPI, MVA, Intel, Cray} & 1 & 0 & 1 & 0 & 0 & 0 & 0 & 0 & 0 & 0 \\%
{OMPI, Intel, Tianhe} & 1 & 0 & 0 & 0 & 0 & 0 & 0 & 0 & 0 & 1 \\%
{MPICH, OMPI, mpi4py} & 1 & 1 & 0 & 0 & 0 & 0 & 0 & 0 & 0 & 0 \\%
{OMPI, Cray, NEC} & 1 & 0 & 0 & 0 & 0 & 0 & 0 & 0 & 1 & 0 \\%
{MPICH, OMPI, Intel, madmpi} & 1 & 1 & 0 & 0 & 0 & 0 & 0 & 0 & 0 & 0 \\%
{MPICH, OMPI, MVA, Intel, Cray, Fujitsu, NEC} & 1 & 0 & 0 & 0 & 0 & 1 & 0 & 0 & 0 & 0 \\%
{OMPI, Intel, HPE, NEC} & 1 & 0 & 0 & 0 & 0 & 1 & 0 & 0 & 0 & 0 \\%
{MVA, Sunway} & 1 & 0 & 0 & 0 & 0 & 0 & 0 & 0 & 0 & 1 \\%
{OMPI, Intel, ParaStation MPI} & 1 & 0 & 1 & 0 & 0 & 0 & 0 & 0 & 0 & 0 \\%
{MPICH, IBM} & 1 & 0 & 0 & 0 & 0 & 0 & 0 & 1 & 0 & 0 \\%
{MVA, Intel, Cray, IBM} & 1 & 0 & 0 & 0 & 0 & 0 & 0 & 0 & 1 & 0 \\%
{MPICH, MVA, Intel, ParaStation MPI} & 1 & 0 & 1 & 0 & 0 & 0 & 0 & 0 & 0 & 0 \\%
{MPICH, OMPI, Intel, IBM, PGI MPI} & 1 & 0 & 0 & 0 & 0 & 0 & 0 & 0 & 1 & 0 \\%
{MPICH, OMPI, Intel, Cray, Sun MPI} & 1 & 0 & 1 & 0 & 0 & 0 & 0 & 0 & 0 & 0 \\%
{OMPI, Cray, IBM} & 1 & 0 & 0 & 0 & 0 & 0 & 0 & 0 & 1 & 0 \\%
{Intel, HPE} & 1 & 0 & 0 & 0 & 0 & 0 & 0 & 0 & 1 & 0 \\%
{OMPI, Sunway} & 1 & 0 & 0 & 0 & 0 & 0 & 0 & 0 & 0 & 1 \\%
{MPICH, OMPI, MVA, Intel, Cray, IBM, MS, Adaptive MPI} & 1 & 0 & 0 & 0 & 0 & 0 & 0 & 0 & 1 & 0 \\%
{MPICH, OMPI, Intel, IBM, BULLMPI} & 1 & 1 & 0 & 0 & 0 & 0 & 0 & 0 & 0 & 0 \\%
{OMPI, MS} & 1 & 0 & 0 & 1 & 0 & 0 & 0 & 0 & 0 & 0 \\%
{MPICH, OMPI, Intel, MPC} & 1 & 1 & 0 & 0 & 0 & 0 & 0 & 0 & 0 & 0 \\%
{MPICH, OMPI, MVA, Intel, NEC} & 1 & 0 & 1 & 0 & 0 & 0 & 0 & 0 & 0 & 0 \\%
{MPICH, OMPI, Intel, Platform} & 1 & 1 & 0 & 0 & 0 & 0 & 0 & 0 & 0 & 0 \\%
\hline%
\end{longtable}%
\end{landscape}}%
\clearpage%


\subsection{List of other answers}

\begin{report}
\begin{itemize}
\item Europe:France: BULL MPI
\item Europe:France: BULLMPI
\item Europe:France: BullMPI
\item Europe:France: BullX-MPI
\item Europe:France: MadMPI
\item Europe:France: MadMPI
\item Europe:France: Platform
\item Europe:France: SGI MPI
\item Europe:France: madMPI
\item Europe:France: madmpi
\item Europe:France: mpi4py
\item Europe:Germany: ParaStation MPI
\item Europe:Germany: ParaStation MPI
\item Europe:Germany: ParaStation MPI
\item Europe:Germany: ParaStation MPI
\item Europe:Germany: ParaStation MPI
\item Europe:Germany: ParaStation MPI
\item Europe:Germany: ParaStation MPI
\item Europe:Germany: ParaStation MPI
\item Europe:Germany: ParaStation MPI
\item Europe:Germany: ParaStation MPI
\item Europe:Germany: Sun MPI
\item Europe:UK: SGI MPT
\item Japan: SGI MPI
\item Russia: SGI MPI
\item USA: Adaptive MPI
\item USA: PGI MPI

\end{itemize}
\end{report}
