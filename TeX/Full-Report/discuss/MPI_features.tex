\section{MPI Features}

Table~\ref{tab:Q10-ans}, \ref{tab:Q14-ans}, \ref{tab:Q16-ans}, and
\ref{tab:Q17-ans} show the answer distribution for 4 questions out of
30. Note that the numbers in these tables reflect that these questions
allow participants to choose multiple answers. 
The actual number of participants who answered
each question is noted after the slash (/) of the total number
at the end of each table.

\begin{table}[htb]%
\begin{center}%
\caption{How did you learn MPI?}%
\label{tab:Q10-ans}%
\begin{tabular}[t]{l|r}%
\hline%
Multiple Choice & \# Answers \\%
\hline%
Other lectures or tutorials & 436 (23.8\%) \\%
Articles found on Internet & 430 (23.4\%) \\%
MPI standard docs & 323 (17.6\%) \\%
Lecture(s) at school & 292 (15.9\%) \\%
Book(s) & 277 (15.1\%) \\%
Never learned MPI & 26 (1.4\%) \\%
other & 50 (2.7\%) \\%
\hline%
\multicolumn{1}{c}{total} & 1834 / 825 \\%
\hline%
\end{tabular}%
\end{center}%
\end{table}

\begin{table}[htb]%
\begin{center}%
\caption{How do you check MPI specifications when you are writing MPI programs?}%
\label{tab:Q14-ans}%
\begin{tabular}[t]{l|r}%
\hline%
Multiple Choice & \# Answers \\%
\hline%
Online docs. & 570 (30.1\%) \\%
Internet & 560 (29.6\%) \\%
MPI standard docs. & 424 (22.4\%) \\%
I ask colleagues & 185 (9.8\%) \\%
I read book(s) & 102 (5.4\%) \\%
I know all MPI routines & 43 (2.3\%) \\%
other & 11 (0.6\%) \\%
\hline%
\multicolumn{1}{c}{total} & 1895 /824 \\%
\hline%
\end{tabular}%
\end{center}%
\end{table}


\begin{table}[htb]%
\begin{center}
\caption{Which MPI features have you never heard of?}%
\label{tab:Q16-ans}%
\begin{tabular}[t]{l|r}%
\hline%
Multiple Choice & \# Answers \\%
\hline%
PMPI interface & 457 (24.5\%) \\%
Persistent comm. & 423 (22.7\%) \\%
Dynamic process creation & 380 (20.4\%) \\%
One-sided comm. & 128 (6.9\%) \\%
Communicator operations & 123 (6.6\%) \\%
MPI datatypes & 90 (4.8\%) \\%
Point-to-point comm. & 89 (4.8\%) \\%
Collective comm. & 86 (4.6\%) \\%
w/ OpenMP (multithread) & 86 (4.6\%) \\%
\hline%
\multicolumn{1}{c}{total} & 1,862 / 691 \\%
\hline%
\end{tabular}%
\end{center}%
\end{table}

\begin{table}[htb]%
\begin{center}
\caption{What aspects of the MPI standard do you use in your program in its current form?}%
\label{tab:Q17-ans}%
\begin{tabular}[t]{l|r}%
\hline%
Multiple Choice & \# Answers \\%
\hline%
Collective comm. & 729 (22.5\%) \\%
Point-to-point comm. & 701 (21.6\%) \\%
MPI datatypes & 512 (15.8\%) \\%
w/ OpenMP (multithread) & 472 (14.6\%) \\%
Communicator operations & 406 (12.5\%) \\%
One-sided comm. & 223 (6.9\%) \\%
PMPI interface & 66 (2.0\%) \\%
Persistent comm. & 61 (1.9\%) \\%
Dynamic process creation & 50 (1.5\%) \\%
other & 20 (0.6\%) \\%
\hline%
\multicolumn{1}{c}{total} & 3,240 / 813 \\%
\hline%
\end{tabular}%
\end{center}
\end{table}

Table~\ref{tab:Q10-ans} shows the result of the question asking ``how
did you learn MPI.'' It is a bit surprising that 40\% (323/825) of
participants refer to the MPI standard. According to the original
data (omitted due to the space limit), one quarter of participants
learned MPI via the internet and/or online documents.

The result of the question asking ``How do you check MPI
specifications when writing MPI programs'' is shown in
Table~\ref{tab:Q14-ans}. More than half of participants refer to the
MPI standard. It is very natural for MPI users to check MPI
specification by referring the MPI standard more often than reading it
for learning. Most notably, the percentage of participants reading
online documents and/or referring the internet occupies 46\% (original
data).

Referring Table~\ref{tab:Q16-ans} and \ref{tab:Q17-ans}, the MPI
features can be categorized into two groups: well-known ones (P2P,
Collectives, and so on) and little-known ones (PMPI, Persistent and
Dynamic process). Surprisingly, this tendency is almost independent
from countries/regions of the participants.
(Figure \ref{fig:Q17} shows the percentages of answers over the
countries/regions having more than 50 answers.)

\comment{
Table~\ref{tab:Q16-ans} shows the answers of the question asking
``which MPI features have you never heard of?'' The answers of ``PMPI
interface,'' and ``Persistent communication,'' and ``Dynamic process
creation'' occupies almost 70\%.  The percentage of participants chose
one or some of them is 64\%.
Table~\ref{tab:Q17-ans} shows the answers of the question asking
``what aspects of the MPI standard do you use in your program in its
current form?'' This question asks the MPI features used by
participants, not if they know or not. Again, there are large gap
between those three features and the other features.
All those features was already standardized in MPI 2.2 which was
release in 2009. Despite the 10-year appearance, those features are
not well-known.
It is very interesting that the regional differences of the question
asking using MPI aspects over the
countries and regions having more
than 50 answers
(Table~\ref{tab:Q17-ans}) are very small as shown in
Figure~\ref{fig:Q17}.
}

\begin{figure}[htb]
\begin{minipage}{0.8\hsize}
\begin{center}
\includegraphics[width=7cm]{../pdfs/Q17.pdf}
\vspace{-10mm}
\caption{What aspects of the MPI standard do you use in your program in its current form?}%
\label{fig:Q17}
\end{center}
\end{minipage}
\end{figure}

\comment{
On the question asking ``how did you learn MPI?,'' one quater of
participants answered ``internet,'' ``not learned,'' and ``other.'' Many
participants who chose ``other'' said ``learned from existing
code,''  ``learned by doing,'' etc.
\comment{Table~\ref{tab:Q14-ans} shows the result of asking ``how do you check
MPI specifications when you are writing MPI programs?'' }
When we ask ``how do you check
MPI specifications when you are writing MPI programs?,'' 45\% of
participants chose the answers only from ``Online Docs,''
``Internet,'' and ``Asking colleagues.''
30\% of participants chose ``too many routines,'' ``too complicated,''
and/or ``I have nobody to ask,'' when the question ``what are your
obstacles to mastering MPI?'' was given.
}
\comment{
\begin{table}[htb]%
\begin{center}%
\small
\caption{How do you check MPI specifications when you are writing MPI programs?}%
\label{tab:Q14-ans}%
\begin{tabular}{l|r}%
\hline%
Multiple Choice & \# Answers \\%
\hline%
{\small I read online documents (such as man pages).} & 570 (30.1\%) \\%
{\small I search the Internet} & 560 (29.6\%) \\%
{\small I read the MPI Standard document}  & 424 (22.4\%) \\%
I ask colleagues. & 185 (9.8\%) \\%
{\small I read book(s) (except the MPI standard).} & 102 (5.4\%) \\%
I know almost all MPI routines. & 43 (2.3\%) \\%
other & 11 (0.6\%) \\%
\hline%
\multicolumn{1}{c}{total} & 1,895 / 824 \\%
\hline%
\end{tabular}%
\end{center}%
\end{table}%
}

Among the lesser-known MPI features, the persistent communication is
one of the important directions being discussed on the MPI Forum\cite{mpi-forum}.
The persistent communication can give implementors room for
optimizing not only P2P but also collective communication
performance.
All those little-known MPI features already appeared in MPI 2.2
released in 2009. Despite the 10-year appearance, those features
fail to be widely accepted.  Why?

One possible answer to this question may come from the survey results
asking participants ``how did you learn MPI''
(Table~\ref{tab:Q10-ans}) and ``how do you check
MPI specifications when you are writing MPI programs''
(Table~\ref{tab:Q14-ans}).
Here a significant number of participants refer to the internet
and/or online documents. These are handy and allow users to get
required information on the fly. However,
these online medias can only be retrieved by
search. To search something, a clue or some keywords must be given.
Say someone wants to search something he/she does not know; how can he/she
find the appropriate key words to obtain the right information?
This could be the responsibility of the information providers. For example,
there is no {\tt See Also} link from the man page of {\tt MPI\_Irecv}
to the corresponding persistent routines in many MPI implementations.

Contrastingly, traditional medias such as books, lectures, and tutorials can be
systematic and complete, but not handy.  Once you learned MPI via some
of those traditional medias, you may think that you already know
MPI. Unfortunately, the MPI standard is being updated by MPI Forum. It is
very hard to keep your MPI knowledge up-to-date, especially if MPI is not
your major concern.

The other point we noticed is that many people who chose the ``other''
answer to the question asking ``how did you learn MPI'' said
``reading existing code,'' ``learn by doing,'' ``reverse
engineering''(!),  and so on. Taking a look at existing code found on
internet might be the way of learning programing nowadays. However,
the underlying rationale of every MPI feature is never simpler than
any other sequential program. It is very likely that novice
programmers writing MPI programs in this way encounter difficulties.

