\item Europe:France: I think the MPI standard should evolve more quickly, and drop backward compatibility when necessary
\item Europe:Germany: Would be nice, but sometimes one has to move forward.
\item Europe:Germany: compatibility for a typical "PhD completion" window is a must
\item Europe:Germany: compatibility is important, but should be sacrificed if necessary to implement important changes, e.g. to support 64-bit ints for message lengths
\item Europe:UK: Both "I prefer to have new API for better performance" and "API should be clearly versioned"
\item Europe:UK: The weasel response - I want the code I use to be backward compatible, but don't mind if other bits break!
\item Europe:others: Comnpatibility is important, but it is okay to delegate old API to a different module/header
\item Europe:others: Compatibility is helpful because of the large codebase.
\item Europe:others: Hard question to answer either or on. It should be clearly versioned, mainly backward compatible unless impossible by important future performance and/or interop reasons. Clear enough?;-)
\item Europe:others: The gist of this survey is that you want to know if the API should be changed. The existing API is a good fit for existing functionality: it is a bit clunky, but the standard is clear, and it is easy to wrap in clean C++. The main problem with the API is that only a small part of the API is effiniently/correctly implemented on many implementations.
\item Russia: I prefer to have new API for better performance. But now I have small parallel programmes.
